\chapter*{Abstract}
\label{cha:abtract}
\addcontentsline{toc}{chapter}{Abstract}

Le Reconfigurable Intelligent Surfaces (RISs) sono una tecnologia emergente che
promette di rivoluzionare le comunicazioni wireless del futuro, permettendo di modificare
in tempo reale le caratteristiche del canale di comunicazione tra un trasmettitore
e un ricevitore quando non è possibile instaurare una connessione line-of-sight.
Tuttavia, lo studio e la valutazione della fattibilità e delle prestazioni di questa
tecnologia, come per molte altre, richiede l'utilizzo di strumenti di simulazione
sofisticati e sviluppati ad hoc. I framework di simulazione sono essenziali per
lo studio di tecnologie ancora in uno stato embrionale come le RIS, poiché permettono
di testare e confrontare diverse configurazioni e algoritmi in un ambiente controllato
e riproducibile, evitando costosi test sperimentali, in particolar modo per prodotti
non ancora disponibili sul mercato. Talvolta però, la complessità dei modelli simulativi
e la mole di dati da processare richiedono un'elevata potenza di calcolo e risorse
hardware, che possono risultare insufficienti per ottenere risultati in tempi
ragionevoli. In questo contesto, il calcolo parallelo offre la possibilità di
sfruttare al meglio le risorse a disposizione, riducendo i tempi di esecuzione e
migliorando le prestazioni, tuttavia complicando in alcuni casi la portabilità e
la manutenibilità del codice. Questo elaborato si propone di analizzare tre diversi
framework di programmazione parallela confrontando le principali caratteristiche
che li contraddistinguono, nell'ottica di valutarne l'efficacia, la versatilità,
e ove applicabile la difficoltà d'integrazione. I framework presi in considerazione
sono: parallelizzazione su CPU tramite la libreria \textit{libpthread} e su GPU tramite
le librerie \textit{CUDA} e \textit{OpenCL}. Per questo scopo, è stata eseguita l'integrazione
di ciascuno di questi paradigmi nella libreria \textit{CoopeRIS}, framework di simulazione
per le RISs, al fine di valutarne l'impatto sulle prestazioni delle simulazioni.
I risultati ottenuti da questo studio dimostrano pienamente l'efficacia del calcolo
parallelo, in particolare su GPU, nel ridurre i tempi di esecuzione delle citate
simulazioni, essenzialmente indicando un miglioramento pressoché lineare nella
velocità di esecuzione al crescere del numero di thread utilizzati in
riferimento all'implementazione su CPU, e un incremento di quasi due ordini di grandezza
tramite le implementazioni su GPU.