\chapter*{Abstract}
\label{cha:abtract}
\addcontentsline{toc}{chapter}{Abstract}

Il parallelismo è il concetto più rilevante nello sviluppo di software
efficiente e scalabile, soprattutto nei casi in cui questo ha la necessità di processare
una grande mole di dati e richieda un'elevata potenza di calcolo. In questo
contesto, il calcolo parallelo offre la possibilità di sfruttare al meglio le risorse
hardware disponibili, riducendo i tempi di esecuzione e migliorando le
prestazioni, tuttavia complicando in alcuni casi la portabilità e la manutenibilità
del codice. Questo elaborato si propone di analizzare tre diversi framework di
programmazione parallela confrontando le principali caratteristiche che li
contraddistinguono, nell'ottica di valutarne l'efficacia, la versatilità, e ove applicabile
la difficoltà d'integrazione. I framework presi in considerazione sono: parallelizzazione
su CPU tramite la libreria \textit{libpthread} e su GPU tramite le librerie
\textit{CUDA} e \textit{OpenCL}. Per questo scopo, è stata eseguita l'integrazione
di ciascuno di questi paradigmi nella libreria \textit{CoopeRIS}, framework di
simulazione per le Reconfigurable Intelligent Surfaces, al fine di valutarne l'impatto
sulle prestazioni delle simulazioni. I risultati ottenuti da questo studio
dimostrano pienamente l'efficacia del calcolo parallelo, in particolare su GPU,
nel ridurre i tempi di esecuzione delle citate simulazioni, essenzialmente indicando
un miglioramento pressoché lineare nella velocità di esecuzione al crescere del numero
di thread o core utilizzati in riferimento all'implementazione su CPU, e un
incremento di quasi due ordini di grandezza tramite le implementazioni su GPU.