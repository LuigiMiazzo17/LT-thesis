\chapter{Conclusioni}
\label{cha:conclusioni}

La libreria \textit{CoopeRIS} ha dimostrato in modo evidente come il calcolo parallelo
possa essere sfruttato per ottimizzare e accelerare la simulazione di sistemi complessi
come le Reconfigurable Intelligent Surfaces (RIS). Questo elaborato ha permesso
di approfondire le caratteristiche distintive di questa tecnologia innovativa e
di evidenziare le potenzialità dell'approccio parallelo. Nel corso del lavoro,
sono state analizzate le principali differenze tra un approccio di
parallelizzazione basato su CPU e uno su GPU, mettendo in luce vantaggi e svantaggi
di entrambi. Sebbene il processo di sviluppo e ottimizzazione abbia incontrato
alcune limitazioni e criticità, i risultati ottenuti sono stati soddisfacenti e hanno
raggiunto gli obiettivi prefissati. Il progetto ha inoltre consentito di
acquisire nuove competenze e conoscenze, sottolineando le potenzialità del calcolo
parallelo e della programmazione su GPU. Nonostante i progressi compiuti, questo
lavoro rappresenta solo un punto di partenza per ulteriori sviluppi e miglioramenti.
Futuri interventi potrebbero includere l'implementazione di ulteriori micro-ottimizzazioni
o il perfezionamento di altre procedure non ancora ottimizzate. In conclusione,
la libreria \textit{CoopeRIS} si è rivelata un valido caso di studio per l'ottimizzazione
di un sistema complesso e ha perfettamente evidenziato l'efficacia dell'approccio
parallelo nella simulazione di simili sistemi.